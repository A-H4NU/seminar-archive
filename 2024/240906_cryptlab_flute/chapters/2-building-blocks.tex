\documentclass[../240906_cryptlab_flute.tex]{subfiles}

\begin{document}

\subsection{Boolean Representation of Lookup Table}

\begin{frame}{LUT Revisited}
    \def\onlyred{\only<2->{\color{red}}}
    \begin{columns}
        \begin{column}{0.4\textwidth}
            \centering
            \begin{table}[H]
                \centering
                \begin{tabular}{ccc|c}
                    \(x_1\)       & \(x_2\)       & \(x_3\)       & \(y\)         \\ \hline
                    \onlyred\(0\) & \onlyred\(0\) & \onlyred\(0\) & \onlyred\(1\) \\
                    \(0\)         & \(0\)         & \(1\)         & \(0\)         \\
                    \(0\)         & \(1\)         & \(0\)         & \(0\)         \\
                    \onlyred\(0\) & \onlyred\(1\) & \onlyred\(1\) & \onlyred\(1\) \\
                    \(1\)         & \(0\)         & \(0\)         & \(0\)         \\
                    \onlyred\(1\) & \onlyred\(0\) & \onlyred\(1\) & \onlyred\(1\) \\
                    \(1\)         & \(1\)         & \(0\)         & \(0\)         \\
                    \(1\)         & \(1\)         & \(1\)         & \(0\)
                \end{tabular}
            \end{table}
            Example \(3\)-to-\(1\) LUT \(\Table\)
        \end{column}
        \pause
        \begin{column}{0.4\textwidth}
            \small
            \centering
            \only<2>{
                \begin{align*}
                        & (\ol{x_1} \land \ol{x_2} \land \ol{x_3}) \\
                    \lor & (\ol{x_1} \land x_2 \land x_3) \\
                    \lor & (x_1 \land \ol{x_2} \land x_3)
                \end{align*}
                DNF representation of \(\Table\)
            }
            \pause
            \only<3->{%
                \begin{align*}
                        & (\ol{x_1} \land \ol{x_2} \land \ol{x_3}) \\
                    \oplus & (\ol{x_1} \land x_2 \land x_3) \\
                    \oplus & (x_1 \land \ol{x_2} \land x_3)
                \end{align*}
            }
            \only<3>{An equivalent circuit.}
            \pause
            \begin{equation*}
                y
                = \begin{pmatrix} \ol{x_1} \\ \ol{x_1} \\ x_1 \end{pmatrix}
                   \odot \begin{pmatrix} \ol{x_2} \\ x_2 \\ \ol{x_2} \end{pmatrix}
                   \odot \begin{pmatrix} \ol{x_3} \\ x_3 \\ x_3 \end{pmatrix}
            \end{equation*}
            ``Multi-Fan-In Inner Product'' representation
        \end{column}
    \end{columns}
\end{frame}

\subsection{Building Blocks: Multiplication}
\begin{frame}{Multiplication of \(\lang\cdot\rang\)-shared Values}
    \only<2->{\centerline{%
        \(\begin{aligned}[t]
            z &= uv = (\msf{m}_u \oplus \lambda_u)(\msf{m}_v \oplus \lambda_v) \\
            &=
            {\underbrace{\msf{m}_u \msf{m}_v}_\text{public}} \oplus \msf{m}_u \lambda_v
            \oplus \msf{m}_v \lambda_u \oplus {\underbrace{\lambda_u \lambda_v}_{=\lambda_{uv}}}
        \end{aligned}\)
    }}

    \begin{block}{Multiplication of \(\lang\cdot\rang\)-shared Values}
        \begin{description}[Output]
            \ii[Input]
            \(\lang\cdot\rang\)-shares of \(u, v \in \ZZ_2\)
            and \([\cdot]\)-shares of \(\lambda_{uv} \coloneqq \lambda_u \lambda_v\).
            \ii[Output]
            \(\lang\cdot\rang\)-shares of \(uv\).
        \end{description}
        \pause
        \begin{enumerate}
            \ii
            \(P_i\) computes \([z]_i = i \cdot \msf{m}_u \msf{m}_v
            \oplus \msf{m}_u[\lambda_v]_i \oplus \msf{m}_v [\lambda_u]_i \oplus [\lambda_{uv}]_i\).
            \pause
            \ii
            \(P_i\) randomly chooses \([\lambda_z]_i \in \ZZ_2\).
            \ii
            \(P_i\) computes \([\msf{m}_z]_i = [z]_i \oplus [\lambda_z]_i\)
            and sends it to \(P_{i-1}\).
            \ii
            Then, \(P_0\) and \(P_1\) have \(\lang\cdot\rang\)-shares of \(z = uv\).
        \end{enumerate}
        \pause
        With the prepared multiplication triple,
        a \(\lang\cdot\rang\)-share of \(uv\) is calculated with the cost of \ul{two bits in a single round}.
    \end{block}
\end{frame}

\subsection{Building Blocks: Inner Product}
\begin{frame}{Inner Product of \(\lang\cdot\rang\)-shared Values}
    \begin{block}{Inner Product of \(\lang\cdot\rang\)-shared Values}
        \begin{description}[Output]
            \ii[Input]
            \(\lang\cdot\rang\)-shares of \(u^1, \cdots, u^N, v^1 \cdots, v^N \in \ZZ_2\)
            and \([\cdot]\)-shares of \(\lambda_{u^1} \lambda_{v^1},
            \cdots, \lambda_{u^N} \lambda_{v^N}\).
            \ii[Output]
            \(\lang\cdot\rang\)-shares of \(z = \bigoplus_{k=1}^N u^k v^k\).
        \end{description}
        \pause
        \begin{enumerate}
            \ii \(P_i\) computes \([u^k v^k]_i\) for \(k \in [N]\) as before.
            \pause
            \ii \(P_i\) randomly chooses \([\lambda_z]_i \in \ZZ_2\).
            \ii \(P_i\) computes
            \([\msf{m}_z]_i = [\lambda_z]_i \oplus \bigoplus_{k=1}^N [u^k v^k]_i\)
            and sends it to \(P_{1-i}\).
            \ii
            Then, \(P_0\) and \(P_1\) have \(\lang\cdot\rang\)-shares of \(z = \bigoplus_{k=1}^N u^k v^k\).
        \end{enumerate}
        \pause
        With the prepared \(N\) multiplication triples,
        a \(\lang\cdot\rang\)-share of \(\bigoplus_{k=1}^N u^k v^k\)
        is calculated with the cost of \ul{two bits in a single round}.
    \end{block}
\end{frame}

\subsection{Building Blocks: Multi-Fan-In Product}
\begin{frame}{Multi-Fan-In Product of \(\lang\cdot\rang\)-shared Values}
    \begin{block}{Multi-Fan-In Product of \(\lang\cdot\rang\)-shared Values}
        \begin{description}[Output]
            \ii[Input]
            \(\lang\cdot\rang\)-shares of \(u_1, \cdots, u_M \in \ZZ_2\) and \fbox{\(\mrm{???}\)}.
            \ii[Output]
            \(\lang\cdot\rang\)-shares of \(\bigwedge_{j=1}^M u_j\).
        \end{description}
    \end{block}
    \pause
    \begin{exampleblock}{}
        Let \(\mcal{Q}\) be a set of
        \(\lang\cdot\rang\)-shared values (wires).
        We use the following notation:
        \[\textstyle
            \msf{m}_{\mcal{Q}} \triangleq \bigwedge_{u \in \mcal{Q}} \msf{m}_u
            \quad\text{and}\quad
            \lambda_{\mcal{Q}} \triangleq \bigwedge_{u \in \mcal{Q}} \lambda_u.
        \]
    \end{exampleblock}
    \pause
    \begin{exampleblock}{Observation}
        \[
            \bigwedge_{j=1}^M u_j = \bigwedge_{j=1}^M (\msf{m}_{u_j} \oplus \lambda_{u_j})
            = \bigoplus_{\mcal{Q} \subseteq \mcal{I}}
                (\msf{m}_{\mcal{Q}} \cdot \lambda_{\mcal{I} \setminus \mcal{Q}})
        \]
        where \(\mcal{I} \coloneqq \{\,u_1,u_2,\cdots,u_M\,\}\).
    \end{exampleblock}
\end{frame}

\begin{frame}{Multi-Fan-In Product of \(\lang\cdot\rang\)-shared Values}
    % \centerline{\(\displaystyle
    %     z = \bigoplus_{\mcal{Q} \subseteq \mcal{I}}
    %             (\msf{m}_{\mcal{Q}} \cdot \lambda_{\mcal{I} \setminus \mcal{Q}})
    % \)}
    \begin{block}{}
        \begin{description}[Output]
            \ii[Input]
            \(\lang\cdot\rang\)-shares of \(u_1, \cdots, u_M \in \ZZ_2\) and \([\cdot]\)-shares of
            \(\lambda_{\mcal{Q}}\) for each \(\mcal{Q} \subseteq \mcal{I}\)
            with \(|\mcal{Q}| > 1\) where \(\mcal{I} = \{\,u_1, \cdots, u_M\,\}\).
            \ii[Output]
            \(\lang\cdot\rang\)-shares of \(\bigwedge_{j=1}^M u_j\).
        \end{description}
        \pause
        \begin{enumerate}
            \ii
            \(P_i\) computes
            \centerline{\(
                [z]_i = i \cdot \msf{m}_{\mcal{I}}
                \oplus \bigoplus_{\mcal{Q} \subsetneq \mcal{I}} (\msf{m}_{\mcal{Q}}
                \cdot [\lambda_{\mcal{I} \setminus \mcal{Q}}]_i).
            \)}
            \pause
            \ii
            \(P_i\) randomly chooses \([\lambda_z]_i \in \ZZ_2\).
            \ii
            \(P_i\) computes \([\msf{m}_z]_i = [z]_i \oplus [\lambda_z]_i\)
            and sends it to \(P_{1-i}\).
            \ii
            Then, \(P_0\) and \(P_1\) have \(\lang\cdot\rang\)-shares of \(\bigwedge_{j=1}^M u_j\).
        \end{enumerate}
        \pause
        With the prepared \(2^M - M - 1\) multiplication triples,
        a \(\lang\cdot\rang\)-share of \(\bigwedge_{j=1}^M u_j\)
        is calculated with the cost of \ul{two bits in a single round}.
    \end{block}
\end{frame}

\subsection{Building Blocks: Multi-Fan-In Inner Product}
\begin{frame}{Multi-Fan-In Inner Product of \(\lang\cdot\rang\)-shared Values}
    \begin{block}{}
        \begin{description}[Output]
            \ii[Input]
            \(\lang\cdot\rang\)-shares of \(\vec{u}^1, \cdots, \vec{u}^M \in (\ZZ_2)^N\) and
            \([\cdot]\)-shares of \(\lambda_{\mcal{Q}}\) for each \(\mcal{Q} \subseteq \mcal{I}_k\)
            with \(|\mcal{Q}| > 1\) where \(\mcal{I}_k = \{\,\vec{u}^1_k,\cdots,\vec{u}^M_k\,\}\)
            for \(k \in [N]\).
            \ii[Output]
            \(\lang\cdot\rang\)-shares of
            \(\vec{u}^1 \odot \cdots \odot \vec{u}^N = \bigoplus_{k=1}^N \bigwedge_{j=1}^M \vec{u}^j_k\).
        \end{description}
        \pause
        \begin{enumerate}
            \ii
            \(P_i\) computes
            \centerline{\(\displaystyle
                [z]_i = i \cdot \bigoplus_{k=1}^N \msf{m}_{\mcal{I}_k}
                \oplus \bigoplus_{k=1}^N \bigoplus_{\mcal{Q} \subsetneq \mcal{I}_k} (\msf{m}_{\mcal{Q}}
                \cdot [\lambda_{\mcal{I} \setminus \mcal{Q}}]_i).
            \)}
            \pause
            \ii
            \(P_i\) randomly chooses \([\lambda_z]_i \in \ZZ_2\).
            \ii
            \(P_i\) computes \([\msf{m}_z]_i = [z]_i \oplus [\lambda_z]_i\)
            and sends it to \(P_{1-i}\).
            \ii
            Then, \(P_0\) and \(P_1\) have \(\lang\cdot\rang\)-shares of \(\vec{u}^1 \odot \cdots \odot \vec{u}^N\).
        \end{enumerate}
        \pause
        With the prepared \(N(2^M - M - 1)\) multiplication triples,
        a \(\lang\cdot\rang\)-share of \(\bigodot_{j=1}^N \vec{u}^j\)
        is calculated with the cost of \ul{two bits in a single round}.
    \end{block}
\end{frame}

\end{document}
