\documentclass[../240513_msquare_shor.tex]{subfiles}

\begin{document}

\subsection{Overview}

\begin{frame}{Factorization}
    \begin{block}{Problem ``\textsc{Factoring}''}
        \centerline{Given \(n \in \mbb{N}_{>1}\), find a non-trivial divisor \(d\) of \(n\) if \(n\) is composite.}
    \end{block}
    \pause
    \begin{alertblock}{Unsolved Question}
        이 문제가 고전적인 컴퓨터에서 poly-time안에 해결 가능할까?
        즉, 이 문제는 P에 속할까?
    \end{alertblock}
    \begin{exampleblock}{Current Best: General Number Field Sieve}
    \[\mrm{exp}\left( \left(\left(\frac{64}{9}\right)^{1/3} + o(1)\right) (\log n)^{1/3} (\log \log n)^{2/3} \right)\]
    \end{exampleblock}
    \pause
    \begin{exampleblock}{Shor's Algorithm}
        Randomized quantum computing으로 poly-time안에 해결 가능.
        즉, ``Factoring''은 BQP에 속함.
    \end{exampleblock}
\end{frame}

\begin{frame}{Shor's Algorithm: Overview}
    \centerline{\fbox{Given \(n \in \mbb{N}_{>1}\), find a non-trivial divisor \(d\) of \(n\) if \(n\) is composite.}}
    \begin{block}{Procedure of Shor's Algorithm}
        Input: $n \in \mbb{N}_{>1}$\small
        \begin{enumerate}
            \item If $n$ is even or is a prime power, then done.
            \item Pick a random integer $1 < x < n$ and loop.
            \begin{enumerate}
                \setlength\itemsep{0.4em}
                \item If $\gcd(x, n) > 1$, then done.
                \item Calculate $r = \ord_n x$. \comment{where we need quantum computing}
                \item If $r$ is even and $x^{r/2} \not\equiv_n -1$, break the loop.
            \end{enumerate}
            \item $\gcd(x^{r/2}+1, n)$ and $\gcd(x^{r/2}-1, n)$ are non-trivial divisors.
        \end{enumerate}
    \end{block}
    \begin{exampleblock}{}
        Quantum computing은 order를 계산하는 데에만 쓰임!
    \end{exampleblock}
\end{frame}

\subsection{Correctness}
\begin{frame}{Correctness: Part 1}
    \begin{block}{Theorem}
        If \(b \in \ZZ\) satisfies \(b^2 \equiv 1 \pmod{n}\) and \(b \not\equiv_n \pm 1\),
        then \[ 1 < \gcd(b - 1, n), \gcd(b+1, n) < n\text{.} \]
    \end{block}
    \pause
    \begin{exampleblock}{Proof}
        Set \(d = \gcd(b - 1, n)\) and use B\'ezout's identity. \qed
    \end{exampleblock}
    \pause
    \begin{block}{Corollary}
        If \(r = \ord_n x\) is even and \(x^{r/2} \not\equiv -1 \pmod{n}\),
        then \[ 1 < \gcd(x^{r/2} - 1, n), \gcd(x^{r/2}+1, n) < n\text{.} \]
    \end{block}
    \begin{exampleblock}{}
        위 보조정리에 의해 Shor's algorithm (3)의 정확성은 보여졌다!
    \end{exampleblock}
\end{frame}

\begin{frame}{Correctness: Part 2}
    \begin{exampleblock}{}
        이제 (2) Loop가 무한 루프가 아니라는 걸 보이자!
    \end{exampleblock}
    \begin{block}{Definition}
        소수 \(p\)와 자연수 \(n\)에 대해
        \[
            u(p, n) = \max \{\,k \in \ZZ_{\ge 0} \mid p^k~\text{divides}~n\,\}
        \]
        라고 정의하자.
    \end{block}
    \begin{exampleblock}{Example}
        \begin{itemize}
            \ii \(u(2, 24) = 3\), \(u(3, 24) = 1\).
            \ii \(u(2, 30) = u(3, 30) = u(5, 30) = 1\), \(u(7, 30) = 0\).
        \end{itemize}
    \end{exampleblock}
\end{frame}

\begin{frame}{Correctness: Part 2}
    \begin{block}{Lemma}
        소수 \(p\)와 자연수 \(n\), 음이 아닌 정수 \(\alpha\)에 대해,
        \[ \left| \left\{ k \in [n] \:\big|\: u\big( p, \gcd(k, n) \big) = \alpha \right\} \right| = \begin{cases}
            \displaystyle \frac{p-1}{p^{\alpha+1}}n \vspace*{.7em}, & 0 \leq \alpha < u(p, n) \\
            \displaystyle \frac{1}{p^\alpha}n, & \alpha = u(p, n) \\
            0 & \alpha > u(p, n)
        \end{cases} \]
        이다.
    \end{block}
    \pause
    \begin{block}{Lemma}
        n이 짝수이면
        $\displaystyle \left| \left\{ k \in [n] \:\bigg|\: u\left( 2, \frac{n}{\gcd(k, n)} \right) = \alpha \right\} \right| \leq \frac{n}{2}$.
    \end{block}
\end{frame}

\begin{frame}{Correctness: Part 2}
    \begin{block}{Theorem}
        \(n\)이 odd prime power이면 모든 자연수 \(\alpha\)에 대해
        \[
            \Pr_{x \in \ZZ_n^\ast} \big[ u(2, \ord_n x) = \alpha \big] \le \frac{1}{2}
        \]
        이다. (\(\ZZ_n^\ast = \{\,m \in [n] \mid \gcd(m,n)=1\,\}\))
    \end{block}
    \pause
    \begin{exampleblock}{Proof}
        \(n\)이 odd prime power이므로 \(n\)의
        primitive root \(g\)가 존재한다.
        그러면, 각각의 \(x \in \ZZ_n^\ast\)에 대해 \(x \equiv g^k \pmod{n}\)인
        \(k\)가 존재하고 따라서
        \[
            \ord_n x = \frac{\ord_n g}{\gcd(k, \ord_n g)} = \frac{\phi(n)}{\gcd(k, \phi(n))}
        \]
        이다. 이제 전 슬라이드의 lemma를 적용한다. \qed
    \end{exampleblock}
\end{frame}

\begin{frame}{Correctness: Part 2}
    \begin{block}{Theorem}
        \(n = p_1^{a_1} \cdots p_k^{a_k}\)를 \(n\)의 소인수분해라고 하자.
        \(\gcd(x, n) = 1\)인 \(x\)를 하나 고르자.
        \(r = \ord_n x\), \(r_i = \ord_{p_i^{a_i}} x\)라고 하자.
        \(r = \ord_n\)가 짝수라면,
        \[
            x^{r/2} \equiv -1 \pmod{n}
            \:\iff\: u(2, r_1) = \cdots = u(2, r_k) > 0\text{이다.}
        \]
    \end{block}
    \pause
    \begin{exampleblock}{Proof}
        \(r = \mrm{lcm}(r_1, \cdots, r_k)\)이다.
        \(r\)을 \(r_i\)로 나눈 몫을 \(s_i\)라고 하자.
        \[
            x^{r/2} \equiv -1 \pmod{n} \:\iff\:
            \fall i \in [n],\: x^{r_i s_i / 2} = x^{r/2} \equiv -1 \pmod{p_i^{a_i}}
        \]
        만약 어느 \(s_i\)가 짝수라면 \(x^{r_i s_i / 2} \equiv 1 \pmod{p_i^{a_i}}\)가 되므로,
        모든 \(s_i\)는 홀수이고, 따라서 모든 \(r_i\)는 짝수이다.
        \hfill Continued...
    \end{exampleblock}
\end{frame}

\begin{frame}{Correctness: Part 2}
    \begin{exampleblock}{}
        모순을 보이기 위해 \(u(2, r_1) > u(2, r_2)\)라고 가정하자.
        \(r_1 = 2^{\alpha_1} t_1\), \(r_2 = 2^{\alpha_2} t_2\)로 쓰자. (\(t_1, t_2\)는 홀수.)
        그러면 \(\alpha_1 > \alpha_2\)이므로
        \[
            2^{\alpha_1} \mid r = r_2 s_2 = 2^{\alpha_2} t_2 s_2
        \]
        이고, \(t_2\)와 \(s_2\) 모두 홀수이므로 이는 불가능하다.

        \vspace*{1em}
        반대로, \(u(2, r_1) = \cdots = u(2, r_k) > 0\)을 가정하면
        모든 \(i\)에 대해 \(2 \nmid s_i\)이고,
        따라서 \(x^{r/2} = (x^{r_i / 2})^{s_i} \equiv (-1)^{s_i} = -1 \pmod{p_i^{s_i}}\)이다.
        \qed
    \end{exampleblock}
\end{frame}

\begin{frame}{Correctness: Part 2}
    \begin{block}{Lemma}
        \[
            \Pr_{x \in \ZZ_n^\ast} [ \ord_n x~\text{is even} ] \ge \frac{1}{2}
        \]
    \end{block}
    \begin{exampleblock}{Proof}
        일대일 함수
        \begin{align*}
            f \colon \ZZ_n^\ast &\longrightarrow \ZZ_n^\ast \\
            x &\longmapsto -x
        \end{align*}
        는 order가 odd인 원소를 order가 even인 원소로 보낸다. \qed
    \end{exampleblock}
\end{frame}

\begin{frame}{Correctness: Part 2}
    \begin{block}{Theorem}
        \(n\)이 합성수이면,
        \[
            \Pr_{x \in \ZZ_n^\ast} \left[ \ord_n x~\text{is even and}~
            x^{(\ord_n x) / 2} \not\equiv -1 \pmod{n}\right] \ge \frac{1}{4}.
        \]
    \end{block}
    \begin{exampleblock}{}
        이제 Shor's algorithm (2)의 loop가
        평균적으로 \(O(1)\) 안에 끝난다는 사실을 알았음!
    \end{exampleblock}
\end{frame}

\end{document}
