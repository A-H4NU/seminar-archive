\documentclass[../250326_cryptlab_flute_security.tex]{subfiles}

\begin{document}

\subsection{Notations}
\begin{frame}{Notations}
    \begin{itemize}
        \ii
        Let \(\mcal{Q}\) be a nonempty set of
        \(\lang\cdot\rang\)-shared values (wires).
        We use the following notation:
        \[\textstyle
            \msf{m}_{\mcal{Q}} \triangleq \bigwedge_{u \in \mcal{Q}} \msf{m}_u
            \quad\text{and}\quad
            \lambda_{\mcal{Q}} \triangleq \bigwedge_{u \in \mcal{Q}} \lambda_u.
        \]
    \end{itemize}
\end{frame}

\subsection{Lookup Table}
\begin{frame}{Lookup Table}
    \def\onlyred{\only<2->{\color{red}}}
    \begin{block}{LUT (Lookup Table)}
        \(\delta\)-to-\(\sigma\) lookup table \(\Table\) is a function
        \(\Table \colon \{0,1\}^\delta \to \{0,1\}^\sigma\).
    \end{block}
    \begin{columns}
        \begin{column}{0.4\textwidth}
            \centering
            \begin{table}[H]
                \centering
                \begin{tabular}{ccc|c}
                    \(x_1\)       & \(x_2\)       & \(x_3\)       & \(y\)         \\ \hline
                    \onlyred\(0\) & \onlyred\(0\) & \onlyred\(0\) & \onlyred\(1\) \\
                    \(0\)         & \(0\)         & \(1\)         & \(0\)         \\
                    \(0\)         & \(1\)         & \(0\)         & \(0\)         \\
                    \onlyred\(0\) & \onlyred\(1\) & \onlyred\(1\) & \onlyred\(1\) \\
                    \(1\)         & \(0\)         & \(0\)         & \(0\)         \\
                    \onlyred\(1\) & \onlyred\(0\) & \onlyred\(1\) & \onlyred\(1\) \\
                    \(1\)         & \(1\)         & \(0\)         & \(0\)         \\
                    \(1\)         & \(1\)         & \(1\)         & \(0\)
                \end{tabular}
            \end{table}
            Example \(3\)-to-\(1\) LUT \(\Table\)
        \end{column}
        \pause
        \begin{column}{0.4\textwidth}
            \small
            \centering
            \begin{align*}
                & (\ol{x_1} \land \ol{x_2} \land \ol{x_3}) \\
                \oplus & (\ol{x_1} \land x_2 \land x_3) \\
                \oplus & (x_1 \land \ol{x_2} \land x_3)
            \end{align*}
            {DNF Representation of \(\Table\)}
            \pause
            \begin{equation*}
                y
                = \begin{pmatrix} \ol{x_1} \\ \ol{x_1} \\ x_1 \end{pmatrix}
                   \odot \begin{pmatrix} \ol{x_2} \\ x_2 \\ \ol{x_2} \end{pmatrix}
                   \odot \begin{pmatrix} \ol{x_3} \\ x_3 \\ x_3 \end{pmatrix}
            \end{equation*}
            ``Multi-Fan-In Inner Product'' representation
        \end{column}
    \end{columns}
\end{frame}

% \subsection{Oblivious Transfer}
% \begin{frame}{Oblivious Transfer}
%     \begin{block}{Oblivious Transfer}
%         Let \(S\) be a \emph{sender} and \(R\) be a \emph{receiver}.
%         The \(\binom{2}{1}\text{-}\OT_1\) functionality does the following:
%         \centerline{%
%             \begin{tabular}{c|cc}
%                        & \(S\)                  & \(R\)             \\ \hline
%                 Input  & \(a_0, a_1 \in \ZZ_2\) & \(r \in \{0,1\}\) \\
%                 Learns & Nothing                & \(a_r\)
%             \end{tabular}
%         }
%     \end{block}
%     \pause
%     \begin{block}{Random Oblivious Transfer}
%         The \(\binom{2}{1}\text{-}\rOT_1\) functionality does the following:
%         \centerline{%
%         \begin{tabular}{c|cc}
%                    & \(S\)                  & \(R\)             \\ \hline
%             Input  & Nothing                & \(r \in \{0,1\}\) \\
%             Learns & \(a_0, a_1 \in \ZZ_2\) & \(a_r\)
%         \end{tabular}}
%         where \(a_0\) and \(a_1\) are randomly chosen.
%     \end{block}
% \end{frame}

\subsection{Secret Sharing Schemes}
\begin{frame}{\([\cdot]\) Secret Sharing}
    \centerline{\(P_0\), \(P_1\): parties}
    \begin{block}{\([\cdot]\) Secret Sharing}
        A bit \(v \in \ZZ_2\) is said to be \([\cdot]\)-\emph{shared}
        between \(P_0\) and \(P_1\) if \(P_i\) holds \([v]_i\) such that
        \[v = [v]_0 \oplus [v]_1.\]
    \end{block}
\end{frame}

\begin{frame}{Multiplication of \([\cdot]\)-shared Values}
    \begin{block}{Beaver's Multiplication Triple}
        Suppose \(P_0\) and \(P_1\) have \([\cdot]\)-shares of \(u, v \in \ZZ_2\).
        Suppose additionally that they have \([\cdot]\)-shares of \(a, b, c\) where \(c = ab\).
        \begin{enumerate}
            \ii
            \(P_i\) computes \([u \oplus a]_i\) and \([v \oplus b]_i\),
            and sends them to \(P_{1-i}\).
            \ii
            \(P_0\) and \(P_1\) now know \(d \coloneqq u \oplus a\) and \(e \coloneqq v \oplus b\).
            \ii
            \(P_i\) computes \([z]_i = i \cdot de \oplus d[b]_i \oplus e[a]_i \oplus [c]_i\).
        \end{enumerate}
        Then, \([z]_0 \oplus [z]_1 = z = uv\).
        % With the prepared \emph{multiplication triple},
        % a \([\cdot]\)-share of \(uv\) is calculated with the cost of \ul{four bits in a single round}.
    \end{block}

    \begin{exampleblock}{}
        \[\begin{aligned}[t]
            z &= uv = (d \oplus a)(e \oplus b) \\
              &= {\underbrace{de}_\text{public}} \oplus db \oplus ea \oplus \underbrace{ab}_{=c}
        \end{aligned}\]
    \end{exampleblock}
\end{frame}

\begin{frame}{\(\lang \cdot \rang\) Secret Sharing}
    \begin{block}{\(\lang\cdot\rang\) Secret Sharing}
        A bit \(v \in \ZZ_2\) is said to be \(\lang\cdot\rang\)\emph{-shared} if:
        \begin{enumerate}
            \ii
            A value \(\lambda_v \in \ZZ_2\) is \([\cdot]\)-shared between \(P_0\) and \(P_1\).
            \ii
            A value \(\msf{m}_v \in \ZZ_2\) is public.
            \ii
            \(v = \msf{m}_v \oplus \lambda_v\)
        \end{enumerate}

        The \(\lang \cdot \rang\)-share of \(v\) is denoted
        \(\lang v \rang_i = (\msf{m}_v, [\lambda_v]_i)\).
    \end{block}
\end{frame}

\subsection{Simple Multiplication Functionality}

\begin{frame}{Multiplication of \([\cdot]\)-Shared Values}
    We will use the following functionality as black-box.
    In other words, we assume that there is some protocol that securely realizes this functionality.

    \begin{block}{\(\mcal{F}_\msf{AND}\): Multiplication of \([\cdot]\)-Shared Values (Ideal)}
        \begin{description}[Output]
            \ii[Input] \([\cdot]\)-shares of \(u, v \in \ZZ_2\) from \(P_0\) and \(P_1\).
            \ii[Output] \([\cdot]\)-shares of \(uv \in \ZZ_2\) to each \(P_i\)
        \end{description}
        \begin{enumerate}
            \ii
            Recover \(u, v \in \ZZ_2\) from the shares
            and calculate \(z = uv\).
            \ii
            Sample random \([z]_0 \rgets \ZZ_2\).
            \ii
            Send \([z]_0\) to \(P_0\) and \([z]_0 \oplus z\) to \(P_1\).
        \end{enumerate}
    \end{block}

    \begin{exampleblock}{}
        Note that both \([z]_0\) and \([z]_1\) are
        independent from inputs and are uniformly distributed in \(\ZZ_2\).
    \end{exampleblock}
\end{frame}

\end{document}
