\documentclass[../250428_msquare_provable_security.tex]{subfiles}

\begin{document}

\subsection{Keyed Function}

\begin{frame}{Keyed Function}
    \begin{block}{Keyed Function}
        함수
        \[
            f \colon \mcal{K} \times \{0,1\}^n \to \{0,1\}^n
        \]
        을 생각해 보자.\pause
        이 함수의 첫 번째 parameter \(k \in \mcal{K}\)를 고정하여 얻은 새 함수
        \begin{align*}
                f_k\colon \{0,1\}^n &\longrightarrow \{0,1\}^n \\
                x&\longmapsto f(k, x)
        \end{align*}
        는 \(\mcal{F}_n\)의 원소로 볼 수 있을 것이다.
        \pause
        이런 함수 \(f\)를 \textbf{keyed function}이라고 부른다.
    \end{block}
\end{frame}

\begin{frame}{Pseudorandom Function (PRF)}
    Keyed function \(f \colon \mcal{K} \times \{0,1\}^n \to \{0,1\}^n\)가 주어져 있다.

    전과 비슷하게, 어떤 방어자와 (시간이 \textbf{다항 시간}으로 제한된) 어떤 probabilistic algorithm \(\mcal{D}\)가 공격자인
    다음 ``security game''을 생각해 보자.
    \pause
    \begin{enumerate}
        \ii
        방어자는 \(b \in \{0,1\}\)를 uniform하게 고른다.
        \pause
        \ii
        \(b = 0\)이면, 방어자는
        임의의 key \(k \in \mcal{K}\)를 골라 \(f^\ast = f_k\)로 둔다.
        \ii
        \(b = 1\)이면, 방어자는 임의의 \(f^\ast \dgets \mcal{F}_n\)을 고른다.
        \pause
        \ii
        공격자는 여러 번의 ``질의''를 할 수 있다.
        각 질의 \(x_i\)에 대해 방어자는 \(y_i = f^\ast(x_i)\)를 공격자에게 알려준다.
        \pause
        \ii
        모든 질의의 마지막에 공격자는 \(b' \in \{0,1\}\)을 결정한다.
        만약 \(b = b'\)이면 공격자가 이기고, \(b \neq b'\)이면 방어자가 이긴다.
    \end{enumerate}
\end{frame}

\begin{frame}{Pseudorandom Function (PRF)}
    \begin{block}{Advantage of Distinguisher}
        공격자 \(\mcal{D}\)에 대해,
        \begin{align*}
            P_{\msf{re}} &\coloneqq \Pr[k \dgets \mcal{K} \colon 1 \gets \mcal{D}^{f_k}] \\
            P_{\msf{id}} &\coloneqq \Pr[F \dgets \mcal{F}_n \colon 1 \gets \mcal{D}^{F}]
        \end{align*}
        이라고 하면,
        \[
            \msf{Adv}_f(D) = |P_{\msf{id}} - P_{\msf{re}}|
        \]
        라고 하자.
    \end{block}

    \pause
    \[
        (\text{\(\mcal{D}\)가 이길 확률})
        = \frac{1}{2}P_{\msf{id}} + \frac{1}{2}(1-P_{\msf{re}})
        = \frac{1}{2} + \frac{1}{2}(P_{\msf{id}} - P_{\msf{re}})
    \]
    우리는 \(\mcal{D}\)가 이길 확률을 줄이는 게 아니라 정확히 \(1/2\) 근처로 맞춰야 한다.
    (Why?)
\end{frame}

\begin{frame}{Pseudorandom Function (PRF)}
    \begin{block}{Pseudorandom Function (PRF)}
        만약 모든 \(\mcal{D}\)에 대해 \(\msf{Adv}_f(\mcal{D}) \le (\text{negligible function on \(n\)})\) 이하이면
        \(f\)를 \textbf{pseudorandom function (PRF)}라고 부른다.
    \end{block}
    {
        \(g(n)\) is negligible if \(\lim\limits_{n \to \infty} n^k g(n) = 0\) for all \(k > 0\).
    }
\end{frame}

\begin{frame}{One-Time Pad with PRF}
    \(f\)가 pseudorandom function이면,
    각 \(k \in \mcal{K}\)에 대해
    \[
        f_k(0) \parallel f_k(1) \parallel f_k(2) \parallel \cdots
    \]
    는 random sequence과 구별할 수 없다. 그러면...?
    \pause

    \begin{block}{One-Time Pad with PRF}
        \begin{itemize}
            \ii
            \(\msf{Enc}(k, x) = x \oplus (f_k(0) \parallel f_k(1) \parallel f_k(2) \parallel \cdots)\)
            \ii
            \(\msf{Enc}(k, c) = c \oplus (f_k(0) \parallel f_k(1) \parallel f_k(2) \parallel \cdots)\)
        \end{itemize}
    \end{block}

    이런 One-Time Pad with PRF는 더 이상 완전히 안전하지는 않지만,
    기존의 One-Time Pad를 ``안전하게 구현''한다고 할 수 있다.
\end{frame}

\begin{frame}{Pseudorandom Permutation (PRP)}
    \(f_k \in \mcal{P}_n\)인 keyed function \(f \colon \mcal{K} \times \{0,1\}^n \to \{0,1\}^n\)가
    주어져 있다.

    \pause
    \begin{enumerate}
        \ii
        방어자는 \(b \in \{0,1\}\)를 uniform하게 고른다.
        \ii
        \(b = 0\)이면, 방어자는
        임의의 key \(k \in \mcal{K}\)를 골라 \(f^\ast = f_k\)로 둔다.
        \ii
        \(b = 1\)이면, 방어자는 임의의 \(f^\ast \dgets {\color{red}\mcal{P}_n}\)을 고른다.
        \ii
        공격자는 여러 번의 ``질의''를 할 수 있다.
        각 질의 \(x_i\)에 대해 방어자는 \(y_i = f^\ast(x_i)\)를 공격자에게 알려준다.
        \ii
        모든 질의의 마지막에 공격자는 \(b' \in \{0,1\}\)을 결정한다.
        만약 \(b = b'\)이면 공격자가 이기고, \(b \neq b'\)이면 방어자가 이긴다.
    \end{enumerate}

    \begin{block}{Pseudorandom Function (PRF)}
        만약 모든 \(\mcal{D}\)에 대해 \(\msf{Adv}_f(\mcal{D}) \le (\text{negligible function on \(n\)})\) 이하이면
        \(f\)를 \textbf{pseudorandom permutation (PRP)}라고 부른다.
    \end{block}
\end{frame}

\end{document}
